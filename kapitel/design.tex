\chapter{Design}
\section{Target group}
\todo{outlining target group: prospective students; creating persona}
Before starting with the design of the application, a target group has to be defined. This makes developing of the application purpose easier and also helps not to loose the focus during the design process. A very important aspect of defining a target group is also, to develop the application directly to the needs of the future users. In the end this helps to increase the acceptance of the application. 

The main target group of the virtual reality application will be prospective students, who are thinking of taking a diploma at NYP. As mentioned in the introduction, those students mostly come from a secondary school and have finished their O-Levels, which is an entry requirement for polytechnics. The secondary school educated the students on a very widespread basis, and the students can now decide their specialisation. \\
When the students leave their secondary school, the average age is ??. In this age, people are known to be open to new technologies and trends. Also, the people in the target group are in the stage to make independent decisions for themselves without influence of parents. The application will be designed for prospective student, which show an interest in information technology but yet have no experience or knowledge in that field.\\
In the following, a characteristic of a fictive person is displayed. This person stands as a representative for the target group of prospective students:

\paragraph{Characteristics}
\begin{description}
	\item[Name:] Jessica Lee
	\item[Age:] 16
	\item[Place of residence:] Singapore
	\item[Education:] Finished secondary school
	\item[Personal interests:] Dancing, computer games, TV series
	\item[Level of tech experience:] Rich consumer, no development skills
	\item[Expectation:] Wants to experience a fun virtual reality application

	
\end{description}
\section{Main story concept}
To gain the users attention and let them grow interest in the application it is important to have a good main story. The main story will create a bridge between the several IT courses displayed in the application.
\\
The application will start in the so called ``Smart City''. This is a futuristic urban environment in which the user can move around freely.There are five different buildings along the streets. Each building is interactive and represents a course offered by the NYP. By interacting with the buildings the user can enter laboratories, so called ``labs'', and complete different tasks. Every successful completion of a task has an impact to the main story. \\
The main goal of the VR application is to create a delivery drone which will deliver orders from online shops, just like Amazon announced in a video of 2013 (source). This is still a vision of future for now, but it is a good showcase for the user to show them how they can impact the future with IT. During the completion of the labs several challenges will occur. Once the user started programming their first drone, a lot more drones will start flying around in Smart City. To avoid crashing and chaos, the next task is to provide a proper and stable internet connection to the drones, so they can communicate. But the connected drones have their advantages and disadvantages: At first they are flying normally and communicating to each other but then an unknown IoT-Hacker infiltrates a virus into the drone firmware, so they start to misbehave. Now the user has to trace the hacker with basic cyber security concepts. After the hacker was found, more and more inhabitants of Smart City start to use the drones and provide a lot of data. It is now the job of the user to analyse this data and optimise the drones.