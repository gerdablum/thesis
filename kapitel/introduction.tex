\chapter{Introduction and goals}
\todo{show uni dropout rates and reasons from german statistics;
point out missing knowledge about job rules; describe how it should be that less people quit uni}
At one point of every students life, they have to make the decision about the specialization of their education. This decision is mostly due when applying for a course at a higher educational insitute. 
\section{NYP information technology courses}
Nanyang Polytechnic (NYP) is a polytechnic located in Ang Mo Kio, Singapore. The school offers post-secondary education for students who successfully passed the GCE O-Level examination. The O-Level examination is an annually examination mainly for students who have visited a secondary school. \cite{aboutOLevel} \\ 
It allows the students to take courses either on a junior college, a technical institute or a polytechnic. In Singapore, a polytechnic offers students a more practical and industry based education than a junior college which has a more fundamental approach. In general, students study 3 years at a polytechnic until they get their diploma. After getting a diploma, students can continue their education at a university but they can also start to work in the industry. \cite{schoolSystem}\\
NYP offers a wide range of courses in different fields. One of the fields is information technology (IT). This work will focus on the courses offered at the School of Information Technology, an institute of NYP. In the following the courses will be introduced briefly: \cite{nypCourses}

\paragraph{Information Technology}
This course offers students a broad range of the different fields in information technology. It focuses on an interdisciplinary education. By finishing the course, students are prepared to work in different areas, most commonly in software engineering. During the first year, the fundamental topics of information technology are educated. The second year deepens the knowledge in application programming, database management, software engineering, algorithms and several other topics. The third year will be for specialization. The students can choose elective courses in the fields of artificial intelligence, enterprise cloud computing, geospatial and mobile innovation and cybersecurity. As in all courses offered at the school of information technology, there is one practical industry project and an internship in the third year as well.

\paragraph{Infocomm and Security}
This course covers topics in IT infrastructure, network engineering and security and the internet of things (IoT). Students learn how to program, but also how to secure applications in a connected environment. They also learn about managing connected infrastructures. During the first year there are basic classes in developing and infocomm to get a grounded knowledge in information technology. The second year will deepen the knowledge of network engineering, programming, and IT service management. There will be a practical IoT project for the students as well.
In the third year there will be several elective courses in the areas of system and network security, enterprise infrastructure and infocomm solutions.

\paragraph{Cybersecurity and Digital Forensics}
The cybersecurity and digital forensic course focuses on IT security. This includes securing systems and data for unauthorized access as well as tracing criminals in case a security incident happened. After graduating, students can work as security analysts, network penetration tester, security engineer and similar jobs. During the first year students learn fundamental IT skills. The second year offers classes in forensics, network security, operating systems, security standards and more. Students will also complete an applications security and an infosecurity project.
The last year offers specialization in the topics cybersecurity track and cyber forensics track. It is possible to choose crossdisciplinary classes from other IT courses.

\paragraph{Business Intelligence and Analytics}
The business intelligence and analytics course teaches analytics and interpretation of massive amounts of data. Therefore big data technologies, as well as artificiall intelligence is needed. After graduating, it is possible to work as a data or business analyst, a social media strategist or a digital marketing executive. During the first year student will gain basic IT and business statistic knowledge. The second year offers classes in big data management, digital marketing, predictive modelling and similar topics. Students participate on a big data and a business analytics project.
During the third year, students will do a data science project and can choose between several elective classes.

\paragraph{Business and Financial Technology}
This course focuses on information technology in the financial and business sector. After getting a diploma, students can work as IT or financial consultants, financial application specialists or business and financial analysts. From the beginning the focus lies on connecting business and financial topics with information technology. Therefore, in the first year students learn about economics, accounting and consumer banking as well as about basic programming skills. During the second year this knowledge is deepened through classes like software engineering or financial management. Besides the elective classes, the industry project and internship, students in the third year also collaborate in a fintech innovation project.


