\chapter{Introduction and goals}
The purpose of this thesis is to develop a virtual reality (VR) application for students and prospective students to experience different professions in the IT industry. 
The decision for a specialization of a higher education can be very challenging to students because it has a great impact on their future. Schools and universities struggle to provide comprehensive information about courses and their related job roles. This thesis investigates how the usage of VR can support prospective students by allowing them to experience different professions in the IT industry and giving them an overview of courses that are required to obtain a higher-education degree.


%The advantages of using VR to give information about future job roles are the less costs, less time consuming and yet a nearly real experience for users. The objective of this thesis is not only to give a theoretical research but also to develop a prototype of a VR application for prospective students. 

\section{Motivation}
Missing knowledge about future professions can result in high dropout rates, as seen in \cite{UlrichHeublein.Juni2017}: \\ In total, about one third (32\%) left German universities without a degree in 2016. Among the dropouts, the most common reasons were high requirements and difficult exams (30\%). Other reasons were a lack of motivation (17\%) and not enough practical experience during the studies (15\%).
Students  experience a lack of perseverance in their studies, because of missing imagination and information about their future profession. Therefore, motivation lacks occur. It is harder for students to master theoretical parts of their studies. Furthermore, the study shows that researching about the programme before applying for a course decreases dropout rates. \\
In our opinion, many reasons for a university dropout can be avoided. Therefore, the question rises as to how we can provide suitable information about IT professions to prospective students. Moreover, how is it possible to engage students in getting to know their future job roles?

\section{Goals}
There are many ways to learn about IT courses and job roles. For example, it is possible to read articles, watch a video or ask experienced persons. These approaches provide information about IT job roles in a passive way. Prospective students should actively be involved in their future professions and experience the responsibilities and challenges of the job roles themselves. The media described above are not able to give a first hand impression of a job role. \\
In VR, users can experience a computer generated environment from a fist person perspective. They wear a special headset which fully shields them from the real surroundings. It is possible to get a real and immerse experience of the virtual environment. Users feel like they are actually present in the virtual world. \cite{Linowes.2015}\\
With the usage of VR, it is possible to provide a visualization and simulation of future professions to prospective students. Thus, they are able to experience the job roles themselves. This thesis aims to connect the medium VR with the topic of student acquisition. It should help students with getting a better experience for different professions. Universities can bring various fields of IT closer to prospective students.

\section{Approach}
The core target group are prospective students in Singapore, who finished their secondary education and start with a higher-education degree at a polytechnic.\\
The approach is to develop a VR prototype, that introduces and simulates several IT job roles. This application should provide information about responsibilities and daily tasks in an IT job. We pursue a learning-by-doing approach by letting users complete tasks or minigames. The success of the VR prototype depends upon the design, the storyline and the interactivity, not only on the usage of VR itself.\\
To measure whether the application helps students in improving their understanding of IT job roles, we also test the application. We develop a questionnaire to see, how the understanding of IT job roles has changed. The two comparing states are before and after playing the VR application.\\
The goal of this work is not to implement a complete ready-to-use application, that contains all job roles of the IT industry. Instead, the prototype should contain the simulation of at least one IT job role. The experiences made during the implementation process and the results of the user tests build the result of this work and the base for further implementations.

\section{Hypotheses about advantages of VR for prospective students}
VR can create virtual environments in which users experience a stronger feeling of presence compared to other media. Therefore, it is expected that VR helps prospective students to dive into job roles and provide a realistic overview about IT professions. We expect that students will show more interests in IT job roles after playing the VR application than they do before playing. \\
VR is currently a trending topic in IT. Students are able to use technologies, which they can later work with. The idea is to grow interest in the technologies behind VR and encourage students in developing their own application. Compared to an internship, prospective students can profit from a cheaper and less time consuming approach in getting to know a job role. 
\newpage
\section{Overview of the thesis}
In chapter \ref{stateofarts} we give an overview about basic VR technologies and current developments. Common problems, that occur when designing a VR application and their solutions are discussed. Following, in chapter \ref{design}, we design a storyline for the application and evaluate suitable best practices in hardware and software technologies. The next step is the implementation of the prototype VR application in chapter \ref{implementation}. This application should provide information about different IT job roles related to the IT diploma courses at Nanyang Polytechnic. In the following, we design and perform a user study to test the user experience as well as the main hypothesis. The focus is set on how participants' expectations on IT job roles change after playing the VR application. Chapter \ref{testing} sums up the test procedure and results. Finally, we evaluate the whole project under the aspect of fulfilling the purpose of helping prospective students in their understanding of IT job roles in chapter \ref{conclusion}.

