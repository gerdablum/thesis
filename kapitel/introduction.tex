\chapter{Introduction and goals}
\todo{show uni dropout rates and reasons from German statistics;
point out missing knowledge about job rules; describe how it should be that less people quit uni}
The purpose of this thesis is to develop a virtual reality (VR) application for students and prospects to experience different professions in the IT industry. 
The decision about the specialization of the education can be very challenging for prospective students because it has a great impact for their future. Schools and universities struggle to provide comprehensive information about the courses and especially for the job roles connected to the courses. This thesis investigates, how the usage of VR can support prospective students in their decision making.





%The advantages of using VR to give information about future job roles are the less costs, less time consuming and yet a nearly real experience for users. The objective of this thesis is not only to give a theoretical research but also to develop a prototype of a VR application for prospective students. 

\section{Motivation}
A lack of missing information about future professions can result in high drop out rates, as seen in \cite{UlrichHeublein.Juni2017}, a study about dropout rates in Germany and their reasons: \\ In total, about one third (32\%) left German universities without a degree in 2016. Most of the students quit their course because of too high requirements and difficulties in the exams (30\%). The second important reasons were a lack of motivation (17\%) and the missing of practical work and exercises (15\%).
Looking at the second and third most important reasons for a university dropout together, it is with over one third a very large group, but could be avoidable. With missing imagination and information about the profession the students are later able to work in, the endurance might not be very high. As a conclusion, motivation lacks occur and it is harder for the students to get through more theoretical parts of their studies. As well, it is also shown in the study that occupying with the subject before applying for a course results in less amounts of university dropouts. \\
The question is now: How to provide prospective students suitable information for the profession of their desired course? And more important: How is it possible to engage students in getting to know their future job roles?
\section{Goals}
There are many ways to get information about IT courses and IT job roles. For example, it is possible to read articles or brochoures, watch a video or ask experienced persons. All those approaches give information about IT job roles but the way they  provide it is passive. We want prospective students to actively involve in their future IT professions and let them experience the functions and challenges of the job roles themselves. The media described above are not able to provide integrate the user in a first persons view. \\
 VR describes a computer generated 3D environment which can be experienced by the users. Users are wearing a special headset through which they are fully shielded off from the real surroundings. It is possible to get an immerse and real experience of the virtual environment, and users feel as they are actually present in the virtual world. \cite{Linowes.2015}\\
With the usage of VR, it is possible to provide prospective students a visualization of their future professions and they are able to experience the job roles themselves. This thesis aims to connect the medium VR with the topic of student acquisition. It should help students with getting a better experience for the different IT professions. Universities should profit from this research by having a way to bring the various fields of IT closer to prospective students.

\section{Approach}
The core target group are prospective students in Singapore, who finished their secondary education and start with a specialized education at a polytechnic.\\
The approach is to develop a VR application, which introduces and simulates several IT job roles. This application will provide information about the responsibilities and the daily life in a IT job role. The application will also contain user interactions. Users will complete tasks which are connected to the various IT job roles. This is to follow a learning-by-doing approach. The success of the VR prototype application relies on the design, the storyline and the interactivity, not only on the use of VR itself. Therefore the application will be designed with a focus on user interactivity and a good story line.\\
To measure, how much the application helps students in their experience of IT job roles, the application will also be tested. A questionnaire will be developed to see, how the understanding of IT job roles by the participants has changed. The two comparing states will be before and after playing the VR application.

\section{Hypotheses about advantages of VR for prospective students}
\todo{Explain why VR and future professions work well together, point out immersion of VR, name examples of similar VR projects}
Within the VR application prospective students should not only learn about IT professions but also experience the various job roles through a realistic simulation.\\
VR can create virtual environments in which users experience a stronger feeling of presence than compared to other media. Therefore it is expected, that VR helps prospective students to dive into the job roles and provide a realistic as possible overview about the IT professions. It is expected that students will show more interests in IT job roles than they did before playing the VR application. \\
VR is a currently researched topic of information technology. Students are able to use technologies, which they later get into closer touch with. The idea is to arouse interest in the technology behind VR and to make the students want to develop their own application. Prospective students can profit as well from the cheaper and less time consuming approach in getting to know a future job role, compared to an internship. 


\section{Overview of the thesis}
In chapter \ref{stateofarts} there will be a  research in basic VR technologie and current developments. Common problems which occur when designing a VR application as well as their solutions will be described. Following, in chapter \ref{design}, a storyline for the application will be designed, suitable hardware and software will be evaluated. The next step will be to implement a prototype VR application in chapter \ref{implementation}. This application should provide information about the different IT job roles related to the IT diploma courses at Nanyang Polytechnic. After this, the application will be tested by students. The main focus will be set on how their expectations about IT job roles changes after playing the VR application. Chapter \ref{testing} sums up the test procedure and the results. In conclusion, the whole project will be evaluated under the aspect of fulfilling the purpose of helping prospective students in understanding their future job roles in chapter \ref{conclusion}.

