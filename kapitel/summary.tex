\chapter{Conclusion} \label{conclusion}
\section{Summary}
We implemented a protoype VR application which contains a futuristic city and a laboratory, representing the job role of a software engineer. An interface for the remaining job role laboratories was added in the game. We designed the scene for the cyber security specialist job role and scripted a storyboard but we could not implement the scene because it would have been too extensive in the scope of this work. A main storyline for the whole VR game was scripted as well.\\
There were user tests to prove whether the application helps students and prospectives in understanding their future job roles in the IT industry. The outcome of the tests showed positive results. Participants rated their knowledge of the software engineer job role higher after playing the VR prototype. They could imagine more likely to work as a software engineer compared to their answers before playing the application. However the study is based on the subjective perception of the participants and the results would have been more accurate if there was a higher number of test persons. Unfortunately it was not possible to set up a test with a wider and more objective extent. This would have been gone beyond the scope of this work. The study does not meet international academic standards but should give an impression of the effectiveness of the VR application. It is necessary to repeat the study with a different setting once the implementation is finished. During the user tests, we also tested the usability and user guidance of the game. Based on the results we could improve the guidance by adding more textual hints and highlighting interactive items.\\
We must not forget that the objective was to develop a prototype, not a full application. This objective was fulfilled. It is shown through the user test results that the prototype creates higher interest in the software engineer job role. The methods and concepts used for the implementation of this work's prototype can be reused to implement the remaining job role scenes. 

\section{Outlook}
This work shows that it is possible to create a virtual experience which lets users explore real life jobs on first hand. There is no need to rely on passive reports from professionals, or doing a time consuming internship. This project will be continued by other students of NYP. They will design and implement the remaining IT job role scenes. Furthermore the dialogues in the game have no voice over yet. This feature can be added in future as well. The application can be adapted to support  standalone HMDs. This could increase immersion, because the those HMDs have stronger hardware. We expect that the school can present the VR application on open houses to prospective students, once the implementation is finished. \\The idea of using VR to simulate job roles can be extended to other industries. Through applications similar to this work, the interest in job roles with a less positive reputation,  like nursing or crafting, could be increased.\\
Based on the outcome of this work, we can see that VR creates immerse experiences which cannot be created by traditional media. Because VR is currently a highly researched topic, there will be cheaper and more powerful hardware available on the market to create even more immerse virtual experiences. 
