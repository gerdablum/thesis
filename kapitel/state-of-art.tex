\chapter{State of the art}
\section{VR basic concepts}

\subsection{Different dimensions of reality}
\todo{describe different states: reality, augmented reality, augmented virtuality, vr}
Despites VR, there exist other technologies, which serve the purpose to create a virtual world for the user. They differ in the extend of intervention with the reality. \cite{Tham.2018} show a spectrum of different types of technologies and their dimension of reality.\\
The reality describes the real world and using non digital devices to interact with it, such as a steering wheel for driving a car for example. As already mentioned in the indroduction, AR is a related technology to VR. In the spectrum of \cite{Tham.2018} augmented reality covers the second level of reality dimensions. It describes computer generated elements, which are embedded into the reality. A famous example for a AR application is the game PokemonGo (TODO source), in which players search in the real environment for computer generated items, based on a phone's camera and location services. The next level of reality is the augmented virtuality. This preamble describes a virtual environment, in which users can interact through analogue methods.
\subsection{Immersion and storytelling}
\todo{What is immersion and why so important for VR? What describes a good storytelling? Is there anything special when designing a story for a VR application?}
\section{Hardware for VR}
\todo{describe difference between mobile and desktop HMDs, show examples HTC Vive and Google Daydream, maybe also Google Cardboard. What possibilities of interaction do they provide?}
\section{Showcases of VR applications}

\section{Limits of VR}
\todo{At current status, what is VR not able to achieve? What are the challenges VR researchers face currently?}