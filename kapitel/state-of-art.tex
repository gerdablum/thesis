\chapter{State of the art}
In the introduction chapters, a overview and definition of VR was given. The reader was able to get a small insight in the world of virtual reality. The following chapter will deepen the knowledge in VR and related technologies. Some basic concept about virtual reality will be explained. Later on there will be a summary of the latest state of research. Current VR related hardware devices as well as software applications will be presented. It will also be pointed out, of what VR is not capable yet. The aim of this chapter is to get an overview about the topic of VR and the state of research as it is for now.
\section{VR basic concepts}
\todo{covering some history in this chapter too? E.g. explaining the first wave of VR in the 90s}
\cite{Sherman.2019} describe five core concepts for virtual reality.  According to them, this concepts are the key elements when it comes to design a virtual reality. In the following, this concepts will be explained:
\paragraph{Participants} A participant is a user of a VR application, who is actually wearing the HMD and experiencing the virtual world. Every participant has a different perception of the application, so every experience can be seen as a unique experience. This is because the several participants differ from their cultural background, their age, their expectations and their knowledge in technology. Therefore, it is a challenge for developers of VR applications to provide a virtual world in which various users can feel comfortable. The role of the participants is described as the most important role by \cite{Sherman.2019}, because all the experience is happening in their imagination. Developers or creators are only capable of creating the framework around the experience.
\paragraph{Creators} Creators or developers design and implement a VR application for participants. As mentioned, the developers are not able to create the experience of the users. The core aim of their work is to develop the application, which is a collection of programming scripts, concept designs and data. Together with the participants the experience is created. The following chapters design and implementation (TODO ref) will take a look at the VR application from a creator's perspective, whilst chapter (TODO ref) will dive into the different experiences of participants using the VR application.
\paragraph{Virtual world} In general, the term "virtual world" is defined as the content of a virtual reality application.  The virtual world is the skeleton of the virtual experience. Precisely, it is a description of the objects inside the virtual space. TODO schwammig

\paragraph{Immersion} Immersion in general describes, how deeply users connect to a virutal world. They experience something through a medium which they would not be able to experience without it. They see actions and things through a different point of view and can connect to it in some way. The deeper a user can connect to the virtual environment and the objects inside it, the more immerse the VR experience is for them. In chapter \ref{chap:Immersion and storytelling} the term immersion will be explained in more detail.

\paragraph{Interaction} Most of the time, interaction means a manipulation of a computer generated world. However, interaction can also happen in a non artificial environment. An example for this are interactive novels, or text based computer games. In this mediums/media?, computer graphics are not required and yet they provide a form of user interaction. When it comes to VR, interaction also means relocating a viewport inside a virutal world. Participants can move physically in the virtual environment. Another form of interaction is the collaborative interaction. In some VR applications it is not only possible to interact with the virtual world but also to share the space with other participants and interact with them. This can be in a virtual or physical way.
\subsection{Different dimensions of reality}
\todo{describe different states: reality, augmented reality, augmented virtuality, vr}
Despites VR, there exist other technologies, which serve the purpose to create a virtual world for the user. They differ in the extend of intervention with the reality. \cite{Tham.2018} show a spectrum of different types of technologies and their dimension of reality. A visual representation of this spectrum is displayed in \ref{fig:spectrum}.\\
\begin{figure}[h!]
  \includegraphics[width=14cm]{kapitel/spectrum-of-reality.png}
  \centering
  \caption{Graphical representation of the dimensions of reality by 	  \cite{Lovreglio.2018}}
  \label{fig:spectrum}
\end{figure}

The reality describes the real world and using non digital devices to interact with it, such as a steering wheel for driving a car for example. As already mentioned in the indroduction, AR is a related technology to VR. In the spectrum of \cite{Tham.2018} augmented reality covers the second level of reality dimensions. It describes computer generated elements, which are embedded into the reality. A famous example for a AR application is the game PokemonGo (TODO source), in which players search in the real environment for computer generated items, based on a phone's camera and location services. The next level of reality is the augmented virtuality. This preamble describes a virtual environment, in which users can interact through analogue methods. An example for this could be a video chat room where the participants are present in a virtual room but communicate through analogue methods. As seen in figure \ref{fig:spectrum}, augmented reality and augmented virtuality can be summed up with the term \"mixed reality\" which describes the interaction between real life elements and the virtual environment. Finally, in the last stage is the virtual reality in which a user is fully surrounded by a computer generated environment. This environment is most likely a reproduction of a real-life environment. The user is not only surrounded by an artificial reality but can also interact with it. It might appear that the user actually feels present in the virtual reality. In the following, this phenomenon will be explained in more detail.

\subsection{Immersion and storytelling}
\todo{What is immersion and why so important for VR? What describes a good storytelling? Is there anything special when designing a story for a VR application?}
The term immersion often comes up when talking about VR. As mentioned before, this term describes a experience which users would not be able to percieve without the use of a medium. In general, it describes a experience from a different point of view. This could be for example the story of a different character or the exploration of a unknown location. \cite{Tham.2018} \\
When talking about immersion, it is important to distinguish between two types: mental and physical immersion. Mental immersion is when the participant feels deeply emotionally connected to the virtual world. Physical immersion on the other hand means that the user's physical interactions are transferred into the virtual environment with the use of technology. When talking about immersion in books, films or related media, it is often only referred to mental immersion, because these types of media are not able to provide the technology for interacting with their environments. VR on the other hand can create immersion through a physical way as well. That is why VR can provide a higher grade of immersion than other media. \cite{Tham.2018}\\
Another term closely related to immersion is presence. Presence in the context of VR describes a subjective feeling of participants to actually exist in the virtual world. The feeling of presence is a mix between mental and physical immersion. The more presence user feel, the more they accept the virtual world as the reality. To gain the sense of presence, the participant's movements in the reality should match with the movements of the virtual world. With a higher grade of presence, participants act more natural in a virutal environment, such as they would in a real environment. Through this, their task performance increases. This is one reason why presence is an important factor when it comes to developing VR, especially in the field of education and learning. \cite{Slater.1997}\\

Since the terms immersion and presence are explained, it is now the question, how design the story of a VR application in order to gain a maximum grade of immersion. Storytelling is one of the most important elements for mental immersion in a virtual experience. Especially mental immersion can be influenced by the storyline, while physical immersion can be increased through hardware tools. \\
Storytelling in general is the art of telling a narrative to the user in a way that the user is not only consumer but actually feels immersed with the story.\cite{Louden.2018}. Designing a storyline in VR opens a lot of possibilities in engaging the participant in the story. Therefore there are things to consider when it comes to storytelling in VR: \cite{Keane.2018}\\
One thing to bear in mind is, that space and environment in a VR application are more important than in other media, like books or films. The creator designs a world in a 360 degree view, so there is a lot more space to show. The storyline not only happens in front of the users, but can be all around them. Because the user can walk freely around in the virtual world, it is very important to softly guide the user into a certain direction when necessary for the storyline. The guidance should not be conspicuous, but should attract the users curiosity. This can be achieved through light and sound effects or movements. Another aspect to consider is not to overwhelm the user with complex tasks. This could prevent the user in experiencing an immersed virtual world because their full attention would lie on completing the tasks.
\subsection{Interaction: Selection and manipulation in VR}
\todo{different interaction tasks in VR and their common solution: Simple virtual hand, raycasting, go-go, HOMER, scaled world-grab, occlusion; describing ways of movement (walking aroung) solutions in VR}
In order to get an immersive experience, the user of a VR application needs to interact with the environment. When it comes to interaction in VR, several challenges occur. First of all, there is a three dimensional space which has to be covered. Secondly, when users are wearing the HMDs, they cannot see or interact with controllers in the real world. This means, that every hardware controlling device should be mapped into the virtual world. This chapter will describe some typical interactions and their common solutions in virtual reality.\\
Basically, there are two different types of interaction in VR: Modifying the virtual world and moving in the virtual world. When modifying the virtual world, the user should be able to select and manipulate objects. Manipulating includes moving, scaling, changing of look (color, texture) and orientation of an object in the virtual world. When walking around in the virtual world, the user should be able to move freely and make their own decisions in navigation.\\
A challenge when finding solutions to the described interaction problems is, that there do not exist standard solutions, like we have in a screen based 2D context for example. Nevertheless, There are some best practices on how to implement the described requirements in VR. In the following, the most common interaction techniques will be explained according to \cite{Dorner.2013}.

A very straightforward method in selection could be the selection through gaze. The user can select objects by looking at it. This is a very simple method and works without hardware controllers for user input. An analogy in 2D interfaces would be the selection via mouse over. However, this technique does not work well in VR. The problem is, that users accidental select objects every time they look around. A simple and natural exploring of the virtual world would be interrupted by object selection.\\
Another technique is called ray-casting. It works with a virtual virtual ray coming from the users' hand (TODO picture). Therefore, an controller or a hand tracking hardware is needed, which movements are mapped into the virtual world. Every object which collides with the ray is a possible candidate for selection. Most of the time, the object nearest to the user will be selected. ray-casting is an important method in object selection, because it is very straightforward and conforms to the expectations of users (TODO why?). Nevertheless, this method becomes inaccurate when to deal with large distances, because the movements of the hand has to be more accurate the longer the distance gets. \\
The next method, the Go-Go technique overcomes this problem. A virtual arm, mapped with the user's hand movements is displayed in the virtual world just as done with the ray-casting method. The difference of the Go-Go- technique is, that the virtual arm can be lengthened, if the selectable object is out of range. When the virtual arm is lengthened, the user's hand movements are mapped in a smaller range, than when the virtual arm is in the normal position. This mapping is not linear and overcomes the problem of inaccuracy when selecting objects to far away. The technique can be well used for relocating objects, because the user does not have to walk on the same time. However, with this technique, as well as with the ray-casting technique, only objects which are in viewing range can be selected or manipulated. Some objects could be overlapped by other obstacles and cannot be selected. Also, objects in far distance appear very small, which makes manipulation difficult.\\
The HOMER technique deals manipulation of objects in distance. The abbreviations stands for Hand-Centered Object Manipulation Ray-casting. As the names says, it uses techniques of the ray casting for selection. To make the manipulation easier, the object moves to the users hand, once selected. The user can now perform their desired manipulation. After the manipulation is finished, the objects teleports back to its original position. This technique still has the problem of selecting objects which are not in the user's viewing range.\\
\begin{figure}[h!]
  \includegraphics[width=14cm]{kapitel/mini-map.jpg}
  \centering
  \caption{Mini map as part of the WIM technique. By \cite{Arnowitz.2017}}
  \label{fig:minimap}
\end{figure}
An alternative to this method is the world in miniature (WIM) technique. As seen in figure \ref{fig:minimap}, the user gets a miniature map of the virtual world with all selectable objects. The user can now select the desired object in the map and it gets highlighted in the virtual world. With this method, the user looses their ego centred point of view. This can be seen as a decrease in an immerse experience, although it increases usability.

When it comes to moving in the virtual world, the problem of only seeing the virtual world but not the real world becomes more real. There is a danger of physical damage when colliding with real world object while wearing a HMD. There is a conflict between the real world limited space and the unlimited virtual world (TODO source). Therefore some techniques are described also by \cite{Dorner.2013} how to deal with the problems of navigating through virtual worlds:\\
One method in moving around in VR can be the direct manipulation of the camera with a suitable input controller. The user moves the camera with the help of a joystick for example. Another method, which is very common in 3D computer games and therefore conforms highly to the expectations of users, is the walking in the direction of the user's point of view. A disadvantage of this method is, that the user cannot look around while moving. Both described techniques have the disadvantage, that they can cause motion sickness. This phenomenon will be described in detail later.\\
(TODO find source for this) To overcome the motion sickness problem, a very common technique for navigation is teleporting. Teleporting means to translate the user directly to a specific position without a time delay. The technique can be realised in VR by selecting the desired translation point via ray-casting and jumping to the point after selection. This method successfully overcomes the problem of motion sickness, but there is a loss of immersion, since this kind of navigation is not very natural.  \\
The most natural and immerse way of navigating is physical walking. Walking in a virtual environment deals with all the problems of space and the above mentioned conflict between real and virtual world space. Natural walking in general needs more complex hardware for tracking the users movement than navigation with the help of a hand controller. One method of walking in VR is the walking in place method. As the name describes it, the user moves in the virtual world by lifting his legs but they are not actually changing their position in the real world.

\section{Hardware for VR}
\todo{describe difference between mobile and desktop HMDs, show examples HTC Vive and Google Daydream, maybe also Google Cardboard. What possibilities of interaction do they provide?}
\section{Showcases of VR applications}
There are a variety of VR applications available for very different use cases and made with diverse concepts. TODO describe an example for VR exploring without interaction, maybe one game and one educational VR application
\section{Limits of VR}
\todo{At current status, what is VR not able to achieve? What are the challenges VR researchers face currently?}
The idea of VR is not a new invention of the past few years. There has already been a peek in research and also prominence to the public in the late 90's and beginning of 2000. However, the technology did not prevail in the economy, because there were several limitations especially in performance of computation. Therefore only a limited amount of ideas were able to be actually implemented. Another problem at that time were the very large and heavy HMDs (TODO insert image). The user could not wear them for a very long time and it caused neck pain and other physical problems. The devices were also not portable.(TODO source).\\
Looking at VR technologies now, a lot has changed. At first, the performace of our processors have increased rapidly. Secondly the HMDs became a lot smaller and lighter compared to the ones of the late 90's. But yet, we are not at the point, where all research is done -- on the contrary: A lot of challenges still occur when creating a VR application today. In the following, the problem of motion sickness will be explained. It will also be pointed out, what solutions are upcoming in the near future. There is also a lot of research in progress, when it comes to maximise physical immersion. It will be explained, what interactions cannot be covered by today's hardware, and what technologies are in development or testing at current stage.
\subsection{Motion sickness}
\subsection{Hardware}