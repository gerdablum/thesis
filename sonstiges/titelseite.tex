%%%%%%%%%%%%%%%%%%%%%%%%%%%%%%%%%%%%%%%%%%%%%%%%%%%%%%%%%%%%%%%%
% Titelseite basierend auf dem Entwurf von Pascal Schuler vom 01.12.2009
%%%%%%%%%%%%%%%%%%%%%%%%%%%%%%%%%%%%%%%%%%%%%%%%%%%%%%%%%%%%%%%%
\begin{titlepage}
\pagestyle{empty}												
%%%%%%%%%%%%%%%%%%%%%%%%%%%%%%%%%%%%%%%%%%%%%%%%%%%%%%%%%%%%%%%%
% Instituts-/KIT-Logo
%%%%%%%%%%%%%%%%%%%%%%%%%%%%%%%%%%%%%%%%%%%%%%%%%%%%%%%%%%%%%%%%
\hspace*{-1cm}													% gesamte Titelseite um 1cm nach rechts verschieben
\parbox[t][21cm][t]{4cm}										% zwischen den parbox'n darf keine Leerzeile existieren
	{															% sonst werden sie untereinander statt nebeneinander gestellt	
	\begin{center}
		\vspace*{-2cm}\hspace*{-2cm}
		%\includegraphics[scale=1.0]{sonstiges/its_logo.png}		% LOGO des ITS
		
		\vspace*{21cm}\hspace*{0cm}
		%\includegraphics[scale=0.6]{sonstiges/kit_logo.pdf}		% KIT-Logo
	\end{center}
	}
%%%%%%%%%%%%%%%%%%%%%%%%%%%%%%%%%%%%%%%%%%%%%%%%%%%%%%%%%%%%%%%%
% Institutssymbol
%%%%%%%%%%%%%%%%%%%%%%%%%%%%%%%%%%%%%%%%%%%%%%%%%%%%%%%%%%%%%%%%
\parbox[t][22cm][t]{12cm}								% parbox = 2.Spalte mit Text
{																			% [Ausrichtung] [Höhe] [AusrichtenInnen] {Breite}
\begin{flushleft}
	\vspace*{-1.6cm}\hspace*{-1.7cm}			% (2-x)cm = Abstand zwischen ITS-Logo und -Name
	\Large\sffamily
	\parbox[t][3.0cm][t]{13.5cm}
		{
			Hochschule Karlsruhe Technik und Wirtschaft\\
			\large Fakultät für Informatik und Wirtschaftsinformatik~\\
		}
%%%%%%%%%%%%%%%%%%%%%%%%%%%%%%%%%%%%%%%%%%%%%%%%%%%%%%%%%%%%%%%%
% Titel
%%%%%%%%%%%%%%%%%%%%%%%%%%%%%%%%%%%%%%%%%%%%%%%%%%%%%%%%%%%%%%%%

	\vspace{3.5cm}\hspace*{-1.7cm}					% Leerzeile vor vertikaler Abstandsangabe notwendig
	\parbox[t][5cm][t]{13.5cm}								% [Ausrichtung] [Höhe] [AusrichtenInnen] {Breite}
		{
			\LARGE 				
			\centering\bfseries\doctitle  
		}
			
%%%%%%%%%%%%%%%%%%%%%%%%%%%%%%%%%%%%%%%%%%%%%%%%%%%%%%%%%%%%%%%%
% Autor
%%%%%%%%%%%%%%%%%%%%%%%%%%%%%%%%%%%%%%%%%%%%%%%%%%%%%%%%%%%%%%%%

	\vspace{2cm}\hspace*{-1.7cm}						% Leerzeile vor vertikaler Abstandsangabe notwendig
	\parbox[t][3cm][t]{13.5cm}
		{
			\LARGE	
			\centerline{\doctype}
			\centerline{von}
			\centerline{\bfseries\docauthor}
		} 		
\end{flushleft}
%%%%%%%%%%%%%%%%%%%%%%%%%%%%%%%%%%%%%%%%%%%%%%%%%%%%%%%%%%%%%%%%
% Betreuer
%%%%%%%%%%%%%%%%%%%%%%%%%%%%%%%%%%%%%%%%%%%%%%%%%%%%%%%%%%%%%%%%
	% Leerzeile vor vertikaler Abstandsangabe notwendig
	\ifthenelse{\equal{\betreuerII}{}}
	{
		\vspace*{2cm}\hspace*{-1.7cm}
	}
	{
		\vspace*{2cm}\vspace*{-\baselineskip}\hspace*{-1.7cm}
	}
	\parbox[t][1cm][t]{13.5cm}
	{
		\ifthenelse{\equal{\betreuerII}{}}
		{
			\leftline{\Large\sffamily Betreuer:~\bfseries\betreuerI}
		}
		{
		\begin{tabbing}
			\Large\sffamily Betreuer: \= \Large\sffamily\bfseries\betreuerI\\
			\> \Large\sffamily\bfseries\betreuerII
		\end{tabbing}
		\vspace*{-\baselineskip}
		}
		\vspace*{0.5cm}
		\hrule
		\vspace*{0.5cm}
		\leftline{\Large\sffamily \monthword{\month}~\the\year}																		
	}
}
\cleardoublepage

%%%%%%%%%%%%%%%%%%%%%%%%%%%%%%%%%%%%%%%%%%%%%%%%%%%%%%%%%%%%%%%%
%           Versicherung
%%%%%%%%%%%%%%%%%%%%%%%%%%%%%%%%%%%%%%%%%%%%%%%%%%%%%%%%%%%%%%%%
\pagestyle{empty}
\vspace*{0.1cm}
Ich versichere, die Arbeit selbständig verfasst und nur die angegebenen Quellen und Hilfsmittel verwendet zu haben. Die wörtlich oder inhaltlich übernommenen Stellen sind als solche kenntlich gemacht. 
\vspace*{3cm}
\begin{center}
Alina Jaud \doclocation, \docdate
\end{center}
\cleardoublepage

\newpage
\textbf{Zusammenfassung}\\
Diese Arbeit beschäftigt sich mit der Hypothese, ob virtuelle Realität (VR) zukünftigen Studierenden IT Berufe näher bringen kann. Dazu wird eine prototypische Anwendung entwickelt, welche eine Simulation von verschiedenen IT Berufen beinhaltet. Die Berufe sollen spielerisch erkundet werden und die Inhalte in einer Geschichte verpackt sein. In der Arbeit wurde zunächst ein Entwurf der Anwendung gemacht und ein Storyboard entwickelt. Im nächsten Schritt wurde der Beruf des Software Engineer ausgesucht und dessen Berufsbild in der VR Anwendung implementiert. Um den Nutzen der Anwendung und die These zu überprüfen wurde eine Studie durchgeführt. Wir testeten dabei ob sich die Teilnehmerinnen und Teilnehmer nach der Benutzung der VR Anwendung eher vorstellen können als Software Engineer zu arbeiten als vor der Benutzung. Die Ergebnisse der Studie waren positiv. Die Teilnehmerinnen und Teilnehmer zeigten mehr Interesse am Beruf des Software Engineers und fühlten sich besser informiert über das Berufsbild. Darüber hinaus konnte sich ein Großteil der Teilnehmerinnen und Teilnehmer vorstellen, als Software Engineer zu
\vspace*{2cm}
 arbeiten. \\
 \vspace*{0.1cm}
\textbf{Abstract}\\
In this work we examine whether virtual reality (VR) helps prospective students in experiencing and learning about IT professions. Therefore, we develop a prototype VR game which simulates IT job roles. Users can explore the various jobs in the game and solve minigames connected to the IT professions. The information of the game is wrapped in a main story. Before beginning with implementing the game we set the requirements and design a storyboard. Then we pick the job role of the software engineer and implement a prototype VR application which contains 
\end{titlepage}
